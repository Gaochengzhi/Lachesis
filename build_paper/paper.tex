\documentclass[UTF8]{ctexart}
\usepackage{xcolor}
\usepackage{listings}
\usepackage{booktabs}
\usepackage{float}
\usepackage{graphicx}
\usepackage{subfigure}
\usepackage{amsmath}
\usepackage{fancyhdr}
\pagestyle{fancy}
\usepackage{geometry}
\geometry{
 a4paper,
 top=30mm,
 }
\lfoot{}
\usepackage{dirtytalk}
\allowdisplaybreaks
\title{基于lambda演算的函数式语言解释器的设计与实现}
\author{高成志}
\date{\today}
\begin{document}
\maketitle
\newpage
\begin{abstract}
    本文 着重于 函数式语言的一种实现 回顾 计算理论 针对解释器构建的一些细节问题 
\end{abstract}
\newpage
\tableofcontents
\newpage


\section{计算理论基础}

\subsection{从最抽象开始——什么是计算?}

\say{Computation is any type of calculation that includes both arithmetical and
non-arithmetical steps and which follows a well-defined model\cite{wiki1}}


\subsection{lamdba演算}

Lambda演算(也写作$\lambda$ 演算)是数理逻辑中的一个形式系统(formal system),用于表达基于函数式抽象模型和变量绑定与替换应用的计算模型。它是一种通用的计算模型(与图灵机等价),由数学家Alonzo Church 在 1930 年代发明。

Lambda 演算包括构造 lambda 项并对它们执行的归约操作(reduction operations)。在 lambda 演算的最简单形式中,仅使用以下规则构建项:

\begin{table}[]
\begin{tabular}{lll}
\hline
语法  & 名称   & 描述        \\
\hline
x               & 变量    &   标识符名称,用来匹配函数中的某个名字\\
($\lambda$ x.M) & 函数定义 & x为变量名,M为函数体,默认采用波兰表达式。例如自增函数:$\lambda$ x.+ x 1\\
MN              & 函数应用 &函数值在参数前面,如($\lambda$ x . plus x x) y = (plus y y) \\
\hline
\end{tabular}
\end{table} 


\subsubsection{}



\section{解释器架构设计}
\subsection{类型设计}
\lstset{
    numbers=left, 
    numberstyle= \tiny, 
    keywordstyle= \color{ blue!70},
    commentstyle= \color{red!50!green!50!blue!50}, 
    frame=shadowbox, % 阴影效果
    rulesepcolor= \color{ red!20!green!20!blue!20} ,
    escapeinside=``, % 英文分号中可写入中文
    xleftmargin=2em,xrightmargin=2em, aboveskip=1em,
    framexleftmargin=2em
}
\lstset{language=C}
\begin{lstlisting}
#include <iostream>
using namespace std;

int main()
{
    cout<<"hello"<<endl;
    return 0;
}
\end{lstlisting}

\subsection{演算与应用}



\section{工程实现}

\subsection{做好准备工作}

\subsubsection{一个可复用的Makefile}
\subsection{重构——保持调理}

\section{成果展示}

\section{总结与回顾}



\subsection{} 
\begin{thebibliography}{}
    \bibitem{wiki1}
    https://en.wikipedia.org/wiki/Computation
    \bibitem{bib2}
    https://web.phb123.com/city/renk 
\end{thebibliography}{66}
\newpage
\section{附录}
\subsection{声明}
\subsection{main.c}
\lstset{
    breaklines=true,  %代码过长则换行
    %numbers=left, %行号在左侧显示
    numberstyle= \small,%行号字体
    keywordstyle= \color{blue},%关键字颜色
    commentstyle=\color{gray}, %注释颜色
}
\begin{lstlisting}
/**
 * File              : main.c
 * License           : The MIT License (MIT)
 * Author            : Gao Chengzhi <2673730435@qq.com>
 * Date              : 11.02.2022
 * Last Modified Date: 17.02.2022
 * Last Modified By  : Gao Chengzhi <2673730435@qq.com>
 */

#include "../lib/mpc/mpc.h"
#include "headline.h"
#include "lachesis_builtin.h"
#include "lachesis_debug.h"
#include "lachesis_environment.h"
#include "lachesis_object.h"
#include "lachesis_type.h"
#include <readline/history.h>
#include <readline/readline.h>
#include <stdio.h>
#include <stdlib.h>
#include <string.h>

mpc_parser_t* Number;
mpc_parser_t* Symbol;
mpc_parser_t* String;
mpc_parser_t* Comment;
mpc_parser_t* Sexpr;
mpc_parser_t* Qexpr;
mpc_parser_t* Expr;
mpc_parser_t* Lispy;

int main(int argc, char** argv)
{
    /*debug mode switch*/

    check_commandline_argument(&argc, argv);

    Number = mpc_new("number");
    Symbol = mpc_new("symbol");
    String = mpc_new("string");
    Comment = mpc_new("comment");
    Sexpr = mpc_new("sexpr");
    Qexpr = mpc_new("qexpr");
    Expr = mpc_new("expr");
    Lispy = mpc_new("lispy");
    /*Language Definition*/
    mpca_lang(MPCA_LANG_DEFAULT, "                                      \
        number    : /-?[0-9]+/ ;                                        \
        symbol    : /[a-zA-Z0-9_+\\-*\\/\\\\=<>!&]+/;                   \
        string    : /\"(\\\\.|[^\"])*\"/;                               \
        comment   : /;[^\\r\\n]*/;                                      \
        sexpr     : '[' <expr>* ']' ;                                   \
        qexpr     : '{' <expr>* '}' ;                                   \
        expr      : <number> | <symbol> | <string> | <comment>| <sexpr> | <qexpr> ;                     \
        lispy     : /^/ <expr>* /$/ ;                                   \
        ",
        Number, Symbol, String, Comment, Sexpr, Qexpr, Expr,
        Lispy); // notice, can't place space after '\' !

    /*init the environment*/

    lenv* e = lenv_new();
    lenv_builtin_init_list(e);

    if (argc == 1) {

        /*print some welcome words*/
        lec_print_headline();

        while (1) {

            char* input = readline("Lachesis > ");
            /*Attention! readline needs -lreadline argument!*/
            if (strcmp(input, "q") == 0) {
                puts("Bye!");
                exit(0);
            }

            if (strncmp(input, "h ", 2) == 0) {
                print_help(input);
                continue;
            }
            add_history(input);

            mpc_result_t raw;

            if (mpc_parse("<stdin>", input, Lispy, &raw)) {
                LObject* x = lobj_eval(e, lobj_read(raw.output));
                lobj_print_line(x);
                lobj_del(x);
                mpc_ast_delete(raw.output);
            } else {
                mpc_err_print(raw.error);
                mpc_err_delete(raw.error);
            }
            free(input);
        }
    }
    /*file support*/
    if (argc >= 2) {

        for (int i = 1; i < argc; ++i) {

            LObject* args = lobj_add(lobj_sexpr(), lobj_string(argv[i]));

            LObject* x = built_in_load(e, args);

            if (x->type == LOBJ_ERR) {
                lobj_print_line(x);
            }

            lobj_del(x);
        }
    }
    /*before end of code*/
    lenv_del(e);
    mpc_cleanup(7, Number, Symbol, String, Comment, Sexpr, Qexpr, Expr, Lispy);

    return 0;
}


\end{lstlisting}
\end{document}
    
